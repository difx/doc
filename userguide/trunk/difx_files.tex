\section{Description of various files} \label{sec:files}

In the descriptions that follow, the locations of some files is given as {\tt /home/vlbiobs}, meaning the directory {\tt /home/vlbiobs/astronomy/}{\em mmmyy}{\tt /}{\em project} or one of its subdirectories.  
Here {\em mmmyy} is the month and year of the project's observation (i.e., {\tt jan08}) and project is the full project name, with segment, in lower case, such as {\tt bw088n}.
In what follows, the ``software correlator project directory'' (sometimes ``project directory'') refers to the directory from which software correlation is to proceed.
This directory can be created by {\tt getjobs} (\S\ref{sec:getjobs}).
File names beginning with a period (e.g., {\tt .acb}) represent file name extensions, typically (but not always) to job file bases, such as {\tt job121.000} .
Examples of many of the following file types for a particular VLBA correlator job are stashed at \url{http://www.aoc.nrao.edu/~wbrisken/NRAO-DiFX-1.1/} .






% .acb ------------------------------------------------------------------------

\subsection{.acb} \label{sec:acb}

\vspace{-20pt}\hspace{60pt}
\difxoneone

\vspace{7pt}

\noindent
When generation of sniffer output files is not disabled, each {\tt .FITS} file written by {\tt difx2fits} will be accompanied by a corresponding {\tt .acb} file. 
This file contains auto-correlation spectra for each antenna for each source.
In order to minimize the output data size, spectra for the same source will only be repeated once per 15 minutes.
The file contains many concatenated records.
Each record has the spectra for all baseband channels for a particular antenna and has the following format.  
Note that no spaces are allowed within any field.
Values in {\tt typewriter} font without comments are explicit strings that are required.

\begin{center}
\begin{tabular}{l l l l}
\hline
Line(s) & Value & Units & Comments \\
\hline
1 & {\tt timerange:} & & \\
  & {\it MJD}    & integer $\ge 1$    & MJD day number corresponding to line \\
  & {\it start time} & string & e.g., {\tt 13h34m22.6s} \\
  & {\it stop time}  & string & e.g., {\tt 13h34m52.0s} \\
  & {\tt obscode:} & & \\
  & {\it observe code} & string & e.g., MT831 \\
  & {\tt chans:} & & \\
  & {\it n}$_{\mathrm{chan}}$ & $\ge 1$ & number of channels per baseband channel \\
  & {\tt x} & & \\
  & {\it n}$_{\mathrm{BBC}}$ & $\ge 1$ & number of baseband channels \\
\hline
2 & {\tt source:} & & \\
  & {\it source name} & string & e.g., {\tt 0316+413} \\
  & {\tt bandw:} & & \\
  & {\it bandwidth} & MHz & baseband channel bandwidth \\
  & {\tt MHz} & & \\
\hline
3 to 2+$n_{\mathrm{BBC}}$ & {\tt bandfreq:} & & \\
  & {\it frequency} & GHz & band edge (SSLO) frequency of baseband channel \\
  & {\tt GHz polar:} & & \\
  & {\it polarization} & 2 chars & e.g. {\tt RR} or {\tt LL} \\
  & {\tt side:} & & \\
  & {\it sideband} & {\tt U} or {\tt L} & for upper or lower sideband \\
  & {\tt bbchan:} & & \\
  & {\it bbc} & {\tt 0} & Currently not used but needed for conformity \\
\hline
3+$n_{\mathrm{BBC}}$ to & {\it antenna number} & $\ge 1$ & antenna table index \\
2+$n_{\mathrm{BBC}}(n_{\mathrm{chan}}+1)$  & {\it antenna name} & string & \\
  & {\it channel number} & $\ge 1$ & $= \mathrm{chan} + (\mathrm{bbc}-1) \cdot n_{\mathrm{chan}}$ for chan, bbc $\ge 1$ \\
  & {\it amplitude} & $\ge 0.0$ & \\
\hline
\end{tabular}
\end{center}

\noindent
The above are repeated for each auto-correlation spectrum record.
This file can be plotted directly with {\tt plotbp} or handled more automatically with {\tt difxsniff}.






% .apc ------------------------------------------------------------------------

\subsection{.apc} \label{sec:apc}

\vspace{-20pt}\hspace{60pt}
\difxonefive

\vspace{7pt}

\noindent
This file type is nearly identical to the better known {\tt .apd} file; the name acronym refers to Amplitude Phase Channel.
The amplitude, phase, and rate for the brightest channel is determined for each IF for each solution interval.
When generation of sniffer output files is not disabled, each {\tt .FITS} file written by {\tt difx2fits} will be accompanied by a corresponding {\tt .apc} file. 
This file contains {\em channel-based} fringe fit solutions typically every 30 seconds for the entire experiment.
These solutions are not of calibration quality but are sufficient for use in evaluating the data quality.

The first line in the file is the observation code, e.g., {\tt MT831} .

Each subsequent line has the same format with the following fields:

\begin{center}
\begin{tabular}{l l l}
\hline
Key & Units/allowed values & Comments \\
\hline
{\it MJD}           & integer $\ge 1$    & MJD day number corresponding to line \\
{\it hour}          & $\ge 0.0$, $< 24.0$ & hour within day \\
{\it source number} & integer $\ge 1$    & source table index \\
{\it source name}   & string             & name of source; no spaces allowed \\
{\it ant1 number}   & integer $\ge 1$    & antenna table index for first antenna \\
{\it ant2 number}   & integer $\ge 1$    & antenna table index for second antenna \\
{\it ant1 name}     & string             & name of antenna 1; no spaces allowed \\
{\it ant2 name}     & string             & name of antenna 2; no spaces allowed \\
{\it n}$_{\mathrm{BBC}}$  & integer $\ge 1$    & number of baseband channels, $n_{\mathrm{BBC}}$ \\
& & The next four columns are repeated $n_{\mathrm{BBC}}$ times \\
\hline
{\it channel}       & $\ge 1$, $\le n_\mathrm{chan}$            & the strongest channel \\
{\it amplitude}     & $\ge 0.0$          & the amplitude of the peak channel \\
{\it phase}         & degrees            & phase of the peak channel \\
{\it rate}          & Hz                 & the channel phase rate \\
\hline
\end{tabular}
\end{center}






% .apd ------------------------------------------------------------------------

\subsection{.apd} \label{sec:apd}

\vspace{-20pt}\hspace{60pt}
\difxoneone

\vspace{7pt}

\noindent
When generation of sniffer output files is not disabled, each {\tt .FITS} file written by {\tt difx2fits} will be accompanied by a corresponding {\tt .apd} file. 
This file contains Amplitude, Phase, Delay (hence the name) and rate results from fringe fit solutions typically every 30 seconds for the entire experiment.
These solutions are not of calibration quality but are sufficient for use in evaluating the data quality.

The first line in the file is the observation code, e.g., {\tt MT831} .

Each subsequent line has the same format with the following fields:

\begin{center}
\begin{tabular}{l l l}
\hline
Key & Units/allowed values & Comments \\
\hline
{\it MJD}           & integer $\ge 1$    & MJD day number corresponding to line \\
{\it hour}          & $\ge 0.0$, $< 24.0$ & hour within day \\
{\it source number} & integer $\ge 1$    & source table index \\
{\it source name}   & string             & name of source; no spaces allowed \\
{\it ant1 number}   & integer $\ge 1$    & antenna table index for first antenna \\
{\it ant2 number}   & integer $\ge 1$    & antenna table index for second antenna \\
{\it ant1 name}     & string             & name of antenna 1; no spaces allowed \\
{\it ant2 name}     & string             & name of antenna 2; no spaces allowed \\
{\it n}$_{\mathrm{BBC}}$  & integer $\ge 1$    & number of baseband channels, $n_{\mathrm{BBC}}$ \\
& & The next four columns are repeated $n_{\mathrm{BBC}}$ times \\
\hline
{\it delay}         & ns                 & the fringe fit delay \\
{\it amplitude}     & $\ge 0.0$          & the amplitude of fringe fit peak \\
{\it phase}         & degrees            & phase of fringe fit peak \\
{\it rate}          & Hz                 & the fringe fit rate \\
\hline
\end{tabular}
\end{center}








% .binconfig ------------------------------------------------------------------

\subsection{.binconfig} \label{sec:binconfig}

\vspace{-20pt}\hspace{90pt}
\difxonefive

\vspace{7pt}

\noindent
The {\tt .binconfig} file is a file created by the user of DiFX and referenced by the {\tt .input} file to specify pulsar options.
The file uses the standard DiFX input file format and has the following parameters:

\begin{center}
\begin{tabular}{l l l}
\hline
Key & Units/allowed values & Comments \\
\hline
NUM POLYCO FILES      & integer $\ge 1$ & Number of polyco files to read ({\em nPoly}) \\
                      &                 & The next row is duplicated {\em nPoly} times \\
POLYCO FILE {\em p}   & string          & Name of {\em p}$^{th}$ polynomial file \\
NUM PULSAR BINS       & integer $\ge 1$ & Number of pulse bins to create ({\em nBin}) \\
SCRUNCH OUTPUT        & boolean         & Sum weighted bins?  If not, write all bins \\
                      &                 & The next rows are duplicated {\em nBin} times \\
BIN PHASE END {\em b} & float 0.0-1.0   & Pulsar phase where bin ends \\
BIN WEIGHT {\em b}    & float $\ge 0.0$ & Weight to use when scrunching \\
\hline
\end{tabular}
\end{center}

The start of one bin is equal to the end of the previous bin; bins wrap around phase 1.0.
The BIN PHASE END parameters must be listed in ascending phase order.
See Sec.~\ref{sec:pulsars} for example usage of {\tt .binconfig} files.






% cal.vlba --------------------------------------------------------------------

\subsection{cal.vlba} \label{sec:cal}

Monitor data that gets attached to FITS files is extracted by {\tt tsm} into a file called {\em project}{\tt cal.vlba} where {\em project} is the name of the project, i.e., {\tt bg167} or {\tt bc120a}.
A single file contains the monitor data for all antennas for the duration of the project.
The file is left in {\tt /home/vlbiobs} and is compressed with {\tt gzip} after some time to save disk space, resulting in additional file extension {\tt .gz}.
The program {\tt getjobs} (\S\ref{sec:getjobs}) can be used to copy this file from its original location in {\tt /home/vlbiobs} and uncompress the file if needed.
A program called {\tt vlog} (sec \S\ref{sec:vlog}) reads this file and produces files called {\tt flag}, {\tt pcal}, {\tt tsys}, and {\tt weather} in the software correlator project directory.







% .calc -----------------------------------------------------------------------

\subsection{.calc} \label{sec:calc}

The main use of the {\tt .calc} file is to drive the geometric model calculations but this file also serves as a convenient place to store information that is contained in the {\tt .fx} file but not in the {\tt .input file} and is needed for {\tt .FITS} file creation.
In the DiFX system, one {\tt .calc} file is created by {\tt job2difx} (\S\ref{sec:job2difx}) for each {\tt .input} file.
This file is read by {\tt calcif2}) (\S\ref{sec:calcif2}) to produce a tabulated delay model, $u, v, w$ values, and estimates of atmospheric delay contributions.

In brief, the parameters in this file that are relevant for correlation include time, locations and geometries of antennas, pointing of antennas (and hence delay centers) as a function of time and the Earth orientation parameters relevant for the correlator job in question.
Additional parameters that are stuffed into this file include spectral averaging, project name, and information about sources such as calibration code and qualifiers.
In the NRAO application of DiFX, source names are faked in the actual {\tt .input} file in order to allow multiple different configurations for the same source.
A parameter called {\em realname} accompanies each source name in the {\tt .calc} file to correctly populate the source file in {\tt .FITS} file creation.

The syntax of this file is similar to that of the {\tt .input} file.
The file consists entirely of key-value pairs separated by a colon.
The value column is not constrained to start in column 21 as it is for the files used by {\tt mpifxcorr}.
There are five sections in the {\tt .calc} file; these sections are not separated by any explicit mark in the file.

The first section contains values that are fixed for the entire experiment and at all antennas --- all data in this section is scalar.
In the following table, all numbers are assumed to be floating point unless further restricted.
The keys and allowed values in this section are summarized below.
Optional keys are identified with a $\star$.

\begin{center}
\begin{tabular}{l l l}
\hline
Key & Units/allowed values & Comments \\
\hline
JOB ID             & integer $\ge 1$& taken from {\tt .fx} file \\
\Oa{JOB START TIME}     & MJD + fraction & start time of original {\tt .fx} file \\
\Oa{JOB STOP TIME}      & MJD + fraction & end time of original {\tt .fx} file \\
OBSCODE            & string         & observation code assigned to project \\
\Oa{SESSION}            & short string   & session suffix to OBSCODE, e.g., {\tt A} or {\tt BE} \\
\Oa{DIFX VERSION}       & string         & version of correlator, e.g. {\tt DIFX-1.5} \\
\Oa{SUBJOB ID}          & integer $\ge 0$& subjob id assigned by {\tt job2difx} (\S\ref{sec:job2difx}) \\
\Oa{SUBARRAY ID}        & integer $\ge 0$& subarray id assigned by {\tt job2difx} \\
START MJD          & MJD + fraction & start time of this subjob \\
START YEAR         & integer        & calendar year of START MJD \\
START MONTH        & integer        & calendar month of START MJD \\
START DAY          & integer        & day of calendar month of START MJD \\
START HOUR         & integer        & hour of START MJD \\
START MINUTE       & integer        & minute of START MJD \\
START SECOND       & integer        & second of START MJD \\
INCREMENT (SECS)   & integer        & seconds between computed model points ({\em inc}) \\
\Oa{SPECTRAL AVG}       & integer $\ge 1$& number of channels to average in FITS creation \\
\Oa{START CHANNEL}      & integer $\ge 0$& start channel number (before averaging) \\
\Oa{OUTPUT CHANNELS}    & integer $\ge 1$& total number of channels to write to FITS \\
                   & $> 0.0 , < 1.0$& fraction of total channels to write to FITS \\
\Oa{TAPER FUNCTION}     & string         & currently only {\tt UNIFORM} is supported \\
DELAY FILENAME     & string         & filename, including path, of {\tt .delay} file to create \\
UVW FILENAME       & string         & filename, including path, of {\tt .uvw} file to create \\
RATE FILENAME      & string         & filename, including path, of {\tt .rate} file to create \\
IM FILENAME        & string         & filename, including path, of {\tt .im} file to create \\
\hline
\end{tabular}
\end{center}

%See the discussion on spectral selection and averaging in \S\ref{sec:selection} for more information on {\tt SPECTRAL AVG}, {START CHANNEL} and {OUTPUT CHANNELS}.
%Note that all three of these parameters are optional with defaults that result in no averaging and writing of all channels.

The second section contains antenna(telescope) specific information.
After an initial parameter defining the number of telescopes, there are {\em nTelescope} sections (one for each antenna), each with the following six parameters.
Lowercase {\em t} in the table below is used to indicate the telescope index, an integer ranging from 0 to {\em nTelescope} - 1.
Note that in cases where units are provided under the Key column, these units are actually part of the key.

\begin{center}
\begin{tabular}{l l l}
\hline
Key & Units/allowed values & Comments \\
\hline
NUM TELESCOPES               & integer $\ge 1$& number of telescopes ({\em nTelescope}). \\
&& The rows below are duplicated {\em nTelescope} times. \\
\hline
TELESCOPE {\em t} NAME       & string & upper case antenna name abbreviation \\
TELESCOPE {\em t} MOUNT      & string & the mount type: altz, equa, xyew, or xyns \\
TELESCOPE {\em t} OFFSET (m) & meters & axis offset in meters \\
TELESCOPE {\em t} X (m)      & meters & X geocentric coordinate of antenna at date \\
TELESCOPE {\em t} Y (m)      & meters & Y geocentric coordinate of antenna at date \\
TELESCOPE {\em t} Z (m)      & meters & Z geocentric coordinate of antenna at date \\
\Oa{TELESCOPE} {\em t} SHELF & string & shelf location of module to correlate \difxoneone \\
\hline
\end{tabular}
\end{center}

Note that the antenna locations are currently taken from the {\tt .fx} job script written by {\tt cjoggen} and are valid for the date of observation.

The third section contains scan specific information.
Except for one initial line specifying the number of scans, {\em nScan}, this section is composed of nine parameters per scan.
Each parameter is indexed by {\em s} which ranges from 0 to {\em nScan} - 1.

\begin{center}
\begin{tabular}{l l l}
\hline
Key & Units/allowed values & Comments \\
\hline
NUM SCANS              & integer $\ge 1$ & number of scans ({\em nScan}). \\
&& The rows below are duplicated {\em nScan} times. \\
\hline
SCAN {\em s} POINTS    & $\ge 1$         & duration of scan in units of {\em inc} \\
SCAN {\em s} START PT  & integer $\ge 0$ & start time of scan in units of {\em inc} since MJD START \\
SCAN {\em s} SRC NAME  & string          & systematic name to match that used in {\tt .input file} \\
SCAN {\em s} REAL NAME & string          & name to use for source in the FITS output file \\
SCAN {\em s} SRC RA    & radians         & J2000 right ascension\\
SCAN {\em s} SRC DEC   & radians         & J2000 declination \\
SCAN {\em s} CALCODE   & string          & usually uppercase letters \\
SCAN {\em s} QUAL      & integer $\ge 0$ & source qualifier \\
%SCAN {\em s} OVERSAMP  & integer $\ge 1$ & oversample factor for this scan \difxoneone \\
\hline
\end{tabular}
\end{center}

The fourth section contains Earth orientation parameters (EOP).
Except for one initial line specifying the number of days of EOPs, {\em nEOP}, this section is composed of five parameters per day of sampled EOP values.
Each parameter is indexed by {\em e} which ranges from 0 to {\em nEOP} - 1.

\begin{center}
\begin{tabular}{l l l}
\hline
Key & Units/allowed values & Comments \\
\hline
NUM EOP                    & integer $\ge 1$ & number of tabulated EOP values ({\em nEOP}) \\
&& The rows below are duplicated {\em nEOP} times. \\
\hline
EOP {\em e} TIME (MJD)     & MJD + fraction  & time of sample; fraction almost always zero \\
EOP {\em e} TAI\_UTC (sec) & integer seconds & leap seconds accrued at time of job start \\
EOP {\em e} UT1\_UTC (sec) & seconds         & UT1 - UTC \\
EOP {\em e} XPOLE (arcsec) & arc seconds     & X coordinate of polar offset \\
EOP {\em e} YPOLE (arcsec) & arc seconds     & Y coordinate of polar offset \\
\hline
\end{tabular}
\end{center}

The final (completely optional) section has a table for positions and velocites of spacecraft.
Each spacecraft is indexed by {\em s} and each row thereof by {\em r}.

\begin{center}
\begin{tabular}{l l l}
\hline
Key & Units/allowed values & Comments \\
\hline
\Oa{NUM SPACECRAFT}             & integer $\ge 0$ & number of spacecraft ({\em nSpacecraft}) \\
&& Everything below is duplicated {\em nSpacecraft} times. \\
\hline
SPACECRAFT {\em s} NAME    & string          & name of spacecraft \\
SPACECRAFT {\em s} ROWS    & integer $\ge 1$ & number of data rows, {\em nRow}$_s$ for spacecraft {\em s} \\
&& The row below is repeated {\em nRow}$_s$ times. \\
\hline
SPACECRAFT {\em s} ROW {\em r} & 7 numbers & tabulated data --- see below \\
\hline
\end{tabular}
\end{center}

Each data vector of data consists of seven double precision values: time (mjd), $x$, $y$, and $z$ (meters), and $\dot{x}$, $\dot{y}$, and $\dot{z}$ (meters per second).
These values should be separated by spaces.






% .delay ----------------------------------------------------------------------

\subsection{.delay} \label{sec:delay}

The {\tt .delay} files contain tabulated interferometer model delays for each antenna for an entire DiFX job.
This file type is typically produced by {\tt calcif2} within DiFX.
Note that the values of the delays in this file have the opposite sign as compared to those generated by CALC and those stored in {\tt .FITS} files, that is, a more positive delay implies ``closer to the source''; negative delays are behind the Earth center, and hence for ground-based antennas are below the horizon.

The file consists entirely of key-value pairs separated by a colon.
There are two sections in the {\tt .delay} file; these sections are not separated by any explicit mark in the file.

The first section contains values that are fixed for the entire experiment --- all data in this section is scalar.
In the following table, all numbers are assumed to be floating point unless further restricted.
The keys and allowed values in this section are summarized below:

\begin{center}
\begin{tabular}{l l l}
\hline
Key & Units/allowed values & Comments \\
\hline
START YEAR         & integer        & calendar year of START MJD \\
START MONTH        & integer        & calendar month of START MJD \\
START DAY          & integer        & day of calendar month of START MJD \\
START HOUR         & integer        & hour of START MJD \\
START MINUTE       & integer        & minute of START MJD \\
START SECOND       & integer        & second of START MJD \\
INCREMENT (SECS)   & integer        & seconds between computed model points ({\em inc}) \\
NUM TELESCOPES     & integer        & number of telescopes, {\em nTelescope} with delay data \\
                   &                & the following line is repeated {\em nTelescope} times \\
\hline
TELESCOPE {\em t} NAME: & string    & {\em t} starts at 0 \\
\hline
\end{tabular}
\end{center}

The second section contains the scan-based information.
First is a line indicating the number of scans to follow.
Then for each scan, numbered by {\em s} ranging from 0 to {\em nScan} - 1, there are 3 lines containing information about the scan, including the number of sampled points within that scan, {\em nPoint}.
Finally there are {\em nPoint}$_\mathit{s}$ + 3 lines containing the tabulated delays (in microseconds), numbered $-$1 through {\em nPoints}$_\mathit{s}$ + 1, indexed with {\em p}.
Note that this includes one sample before the start of the scan and at least one after the scan end, allowing for a quadratic interpolation across the entire scan.
This information is summarized in the following table:

\begin{center}
\begin{tabular}{l l l}
\hline
Key & Units/allowed values & Comments \\
\hline
NUM SCANS              & integer $\ge 1$ & number of scans ({\em nScan}). \\
&& The rows below are duplicated {\em nScan} times. \\
\hline
SCAN {\em s} POINTS    & $\ge 1$         & duration of scan in units of {\em inc} ({\em nPoints}$_\mathit{s}$) \\
SCAN {\em s} START PT  & integer $\ge 0$ & start time of scan in units of {\em inc} since MJD START \\
SCAN {\em s} SRC NAME  & string          & systematic name to match that used in {\tt .input file} \\
\hline
RELATIVE INC {\em p}   & array; see below & {\em nPoint}$_\mathit{s}$ + 3 of these lines per scan \\
\hline
\end{tabular}
\end{center}

Like for the {\tt .rate} and {\tt .uvw} files, the values reported in this file extend one full mode increment ({\em inc}) before and after the actual duration of the scan, and hence will overlap in time by 2$\times$ {\em inc} with consecutive scans.

This file is typically produced by {\tt calcif2}.





% .difx/ ----------------------------------------------------------------------

\subsection{.difx/} \label{sec:difx}

The SWIN format visibilities written by {\tt mpifxcorr} are written to a directory with extension {\tt .difx}.
Typically there will be a single file in this directory, but it is possible that the output data will be split into
multiple smaller files if the first output file gets too large or if correlation is continued from a point midway through correlation (feature yet to be implemented).

These files contain visibility data records.
Each record contains the visibility spectrum for one polarization of one baseband channel of one baseline for one integration time.  
Each starts with a text header and is followed by binary data.
The text header uses the typical DiFX parameters format with the ``Key'' starting at the beginning of a text line and ending with a colon, and the value starting in the 21st column of text.
The header rows occur in the following order:

\begin{center}
\begin{tabular}{l l l}
\hline
Key & Units/allowed values & Comments \\
\hline
BASELINE NUM       & integer & $= (a_1+1)*256 + (a_2+1)$ for $a_1, a_2 \ge 1$ \\
MJD                & integer & date of visibility centroid \\
SECONDS            & float   & seconds since beginning of MJD \\
CONFIG INDEX       & $\ge 0$ & index to {\tt .input} file configuration table \\
SOURCE INDEX       & $\ge 0$ & index to {\tt .delay} file scan number \\
FREQ INDEX         & $\ge 0$ & index to {\tt .input} frequency table \\
POLARISATION PAIR  & 2 of ({\tt R}, {\tt L}, {\tt X}, {\tt Y}) & e.g., {\tt RR} or {\tt RL} \\
DATA WEIGHT        & $\ge 0.0$ & data weight for spectrum; typically $\sim 1$ \\
U (METRES)         & meters & $u$ component of baseline vector \\ 
V (METRES)         & meters & $v$ component of baseline vector \\
W (METRES)         & meters & $w$ component of baseline vector \\
\hline
\end{tabular}
\end{center}

Following the end-of-line mark for the last header row begins binary data in the form of (real, imaginary) pairs of 32-bit floating point numbers.
The {\tt .input} file parameter {\tt NUM CHANNELS} indicates the number of
complex values to expect.  
In the case of upper sideband data, the first reported channel is the ``zero frequency'' channel, that is its sky frequency is equal to the value in the frequency table for this spectrum.  
The Nyquist channel is not retained.
For lower sideband data, the last channel is the ``zero frequency'' channel.
That is, in all cases, the spectrum is in order of increasing frequency and the Nyquist channel is excised.







% .difxlog --------------------------------------------------------------------

\subsection{.difxlog} \label{sec:dotdifxlog}
\vspace{-20pt}\hspace{80pt}
\difxonefive

\vspace{7pt}

\noindent
The {\tt difxlog} program (\S\ref{sec:difxlogprogram}) captures {\tt DifxAlertMessage} and {\tt DifxStatusMessage} message types that are sent from an ongoing software correlation process and writes the information contained within to a human readable text file.
One line of text is produced for each received message.
The first five columns contain the date and time in {\em ddd MMM dd hh:mm:ss yyyy} format (e.g., {\tt Wed Apr 22 12:48:41 2009}).
The sixth column contains a word describing the contents of the remainder of the line:  Options are:
\begin{itemize}
\item[] {\tt STATUS} : The status of the process is described
\item[] {\tt WEIGHTS} : The playback weights for each antenna are listed
\item[] {\em other} : This word represents an alert severity level (one of {\tt FATAL}, {\tt SEVERE}, {\tt ERROR}, {\tt WARNING}, {\tt INFO}, {\tt VERBOSE} and {\tt DEBUG}) and is followed by the alert message itself.
\end{itemize}






% $DIFX_MACHINES --------------------------------------------------------------

\subsection{\$DIFX\_MACHINES} \label{sec:difxmachines}

Environment variable {\tt DIFX\_MACHINES} should point to a file containing a list of machines that are to be considered elements of the software correlator.
Program {\tt genmachines} (\S\ref{sec:genmachines}) uses this file and information within a {\tt .input} file to populate the {\tt .machines} file needed by {\tt mpifxcorr}.
Because usually only one node in a cluster has direct access to a particular Mark5 module (or data from that module), the ordering of computer names in the {\tt .machines} file is important.
Rows in the {\tt \$DIFX\_MACHINES} file contain up to three items, the last one being optional.
The first column is the name of the machine.
The second column is the number of processes to schedule on that machine (typically the number of CPU cores).
The third column is a 1 if the machine is a Mark5 unit and 0 otherwise.
If this column is omitted, the machine will be assumed to be a Mark5 unit if the first 5 characters of the computer name are {\tt mark5}, and will be assumed not to be otherwise.
Comments in this file begin with an octothorpe (\#).
Lines with fewer than two columns (after excision of comments) are ignored.







% .dir ------------------------------------------------------------------------

\subsection{.dir} \label{sec:dir}

Reading directory information off Mark5 modules can take a bit of time (measured in minutes usually).
Since the same modules are often accessed multiple times, the directories are cached in {\tt \$MARK5\_DIR\_PATH/} .
In this directory, there will be one file per module that has been used, named {\em VSN}{\tt .dir}, where {\em VSN} is the volume serial number of the module, i.e. NRAO$-$023.
The format of these files is as follows:
The first line contains three fields: {\em VSN}, the number of scans on the module, {\em nScan}, and either {\tt A} or {\tt B} indicating the last bank the module was installed in.
At the end of this line the characters {\em RT} can be added (by hand) which will cause the modules to be accessed using {\em Real-Time} mode which is tolerant of missing or bad disks within the module.
Then there are {\em nScan} rows containing information about each scan, each with 11 columns.
Values are floating point unless otherwise noted.
\begin{center}
\begin{tabular}{l l l}
\hline
Key & Units/allowed values & Comments \\
\hline
Start byte & 64-bit integer bytes & offset of the scan on the Mark5 module \\
Length & 64-bit integer bytes & length of the scan \\
Start day & integer MJD & the modified Julian day of the scan start \\
Start time & integer seconds & the scan start time \\
Frame num & integer & frame number since last second tick \\
Frames per sec & integer & number of frames per second \\
Scan Duration & seconds & the duration of the scan \\
Frame size & integer bytes & the length of one data frame, including headers \\
Frame offset & integer bytes & the offset to the start of the first entire frame \\
Tracks & integer & the number of data tracks \\
Format & integer & 0 for VLBA format, 1 for Mark4 format, 2 for Mark5B \\
Name & string & scan name, usually including the project code and station \\
\hline
\end{tabular}
\end{center}

{\em Note:} The directory format used for DiFX differs from that used by the hardware correlator (found in {\em /home/fxcorr/mark5}) and are not interchangeable.







% .FITS -----------------------------------------------------------------------

\subsection{.FITS} \label{sec:FITS}

The {\tt .FITS} files discussed here are produced by {\tt difx2fits}.
They aim to conform to the same table structures as the FITS-IDI files produced by the VLBA correlator.
The format is described in AIPS Memo 102, ``The FITS Interferometry Data Interchange Format'', however, this memo is a bit out of date and the data structures described are not in exact agreement with those made by the VLBA correlator; in all cases the format of data produced by the VLBA hardware correlator is favored where the two disagree.
The tables in these FITS files have a nearly 1 to 1 relationship with the tables that are seen within AIPS, though their two letter abbreviations differ.
The following tables are produced by {\tt difx2fits}:
\begin{center}
\begin{tabular}{l l}
\hline
Table & Description \\
\hline
AG & The array geometry table \\
SU & The source table \\
AN & The antenna table \\
FR & The frequency table \\
ML & The model table \\
CT & The correlator (eop) table \\
MC & The model components table \\
SO & The spacecraft orbit table \difxoneone \\
UV & The visibility data table \\
FG & The flag table \\
TS & The system temperature table \\
PH & The phase calibration table (pulse cals and state counts) \\
WR & The weather table \\
GN & The gain curve table \\
GM & The pulsar gate model table \difxoneone \\
\hline
\end{tabular}
\end{center}
Not all of these tables will always be written.







% .fitslist -------------------------------------------------------------------

\subsection{.fitslist} \label{sec:fitslist}

\vspace{-20pt}\hspace{85pt}
\difxonefive

\vspace{7pt}

\noindent
A {\tt .fitslist} file is written by {\tt makefits} and contains the entire list of {\tt .FITS} files for the correlator pass.
Due to the different constraints of the correlation process and the FITS-IDI format, the number of resultant FITS files may be greater or less than the number of jobs.
This file type is used by {\tt difxarch} to ensure that all of the correlated output ends up in the archive.
The file is composed of two parts: a header line and one line for each {\tt .FITS} file.
The header line consists of a series of {\em key=value} pairs.  
Each {\em key} and {\em value} must have no whitespace and no whitespace should separate these words from their connecting {\tt =} sign.
While any number of key-value pairs may be specified, the following ones (which are case sensitive) are expected to be present:
\begin{enumerate}
\item {\tt exper} : the name of the experiment, including the segment code
\item {\tt pass} : the name of the correlator pass
\item {\tt jobs} : the name of the {\tt .joblist} file used by {\tt makefits}
\item {\tt mjd} : the modified Julian day when {\tt makefits} created this file
\item {\tt DiFX} : the version name for the DiFX deployment (the value of {\tt \$DIFX\_VERSION} when {\tt vex2difx} was run)
\item {\tt difx2fits} : the version of {\tt difx2fits} that was run
\end{enumerate}
Each additional line contains information for one {\tt .FITS} file of the correlation pass.
These lines contain three fields:
\begin{enumerate}
\item {\em archiveName} : the name of the file that will get injected into the archive (see \S\ref{sec:archive})
\item {\em fileSize} : the size of the file in MB
\item {\em origName} : the name of the file as produced by {\tt difx2fits} (via {\tt makefits})
\end{enumerate}







% .flag -----------------------------------------------------------------------

\subsection{.flag} \label{sec:dotflag}

The program {\tt job2difx} may write a {\tt .flag} file for each {\tt .input} file it creates.
This file is used by {\tt difx2fits} to exclude nonsense baselines that might have been correlated.
This can occur when multiple subarrays are coming and going.
The format of this text file is as follows.
The first line contains an integer, $n$ --- the number of flag lines to follow.
The next $n$ lines each have three numbers: $MJD_1$, $MJD_2$ and $ant$.
The first two floating point numbers determine the time range of the flag in Modified Julian Days.
The last integer number is the antenna number to flag --- a zero-based index corresponding
to the {\tt TELESCOPE} table of the corresponding {\tt .input} file.







% flag ------------------------------------------------------------------------

\subsection{flag} \label{sec:flag}

A file called {\tt flag} is created when program {\tt vlog} operates on the {\tt cal.vlba} file.
This file contains lists of antenna-based flags generated by the on-line system that should be applied to the visibility data.
This file contains two kinds of lines.
Comment lines begin with an octothorpe (\#) and contain no vital information.
Flag lines always consist of exactly 5 fields:
\begin{enumerate}
\item {\em antId} : Station name abbreviation, e.g., {\tt LA}
\item {\em start} : Beginning of flagged period (day of year, including fractional portion)
\item {\em end} : End of flagged period (day of year, including fractional portion)
\item {\em recChan} : Record channel affected; -1 for all record channels, otherwise a zero-based index
\item {\em reason} : Reason for flag, enclosed in single quotes, truncated to 24 characters
\end{enumerate}
The flag rows are sorted first by antenna, and then start time.







% .fx -------------------------------------------------------------------------

\subsection{.fx} \label{sec:fx}

The program {\tt cjobgen} is used to create job scripts (with filename extension {\tt .fx}) for use with the VLBA hardware correlator.
Although the functionality of {\tt cjobgen} will eventually be replaced, it is convenient to use the {\tt .fx} files it creates in the interim as they contain all the information required to calculate a delay model and drive the software correlator.
These files are converted by {\tt job2difx} (\S\ref{sec:job2difx}) to produce {\tt .input} files to control correlation and {\tt .calc} files to drive the delay model generator.
Note that there will not in general be a 1 to 1 relationship between {\tt .fx} files and {\tt .input} or {\tt .calc} files due to combination of multiple passes and splitting due to frequency changes.
The program {\tt getjobs} (\S\ref{sec:getjobs}) can be used to copy these files from their original location in {\tt /home/vlbiobs}.







% .im -------------------------------------------------------------------------

\subsection{.im} \label{sec:im}

\vspace{-20pt}\hspace{60pt}
\difxoneone

\vspace{7pt}

\noindent
The {\tt .im} file contains polynomial models used by {\tt difx2fits} in the creation of {\tt FITS} files.
After a header that is similar to that of a {\tt .rate} file, the contents are organized hierarchically with scan number, sub-scan interval, and antenna number being successively faster-incrementing values.
The keys and allowed values in this section are summarized below:
Note that the values of the delay polynomials in this file have the opposite sign as compared to those generated by CALC and those stored in {\tt .FITS} files.
Keys preceded by $\star$ are optional.
Note that all polynomials are expanded about their {\tt MJD, SEC} start time and use seconds as the unit of time.

\begin{center}
\begin{tabular}{l l l}
\hline
Key & Units/allowed values & Comments \\
\hline
\Oa{CALC SERVER}        & string         & name of the calc server computer used \\
\Oa{CALC PROGRAM}       & integer        & RPC program ID of the calc server used \\
\Oa{CALC VERSION}       & integer        & RPC version ID of the calc server used \\
START MJD          & MJD + fraction & start time of this subjob \\
START YEAR         & integer        & calendar year of START MJD \\
START MONTH        & integer        & calendar month of START MJD \\
START DAY          & integer        & day of calendar month of START MJD \\
START HOUR         & integer        & hour of START MJD \\
START MINUTE       & integer        & minute of START MJD \\
START SECOND       & integer        & second of START MJD \\
POLYNOMIAL ORDER   & 2, 3, 4 or 5   & polynomial order of interferometer model {\em order} \\
INTERVAL (SECS)    & integer        & interval between new polynomial models \\
ABERRATION CORR    & $\left\{\begin{array}{l}\mbox{\tt UNCORRECTED}\\\mbox{\tt APPROXIMATE}\\\mbox{\tt EXACT}\end{array}\right.$ & level of $u, v, w$ aberration correction \\
NUM TELESCOPES               & integer $\ge 1$& number of telescopes ({\em nTelescope}) \\
&& The row below is duplicated {\em nTelescope} times. \\
\hline
TELESCOPE {\em t} NAME       & string & upper case antenna name abbreviation \\
\hline
NUM SCANS          & integer $\ge 1$ & number of scans ({\em nScan}). \\
&& Everything below is duplicated {\em nScan} times. \\
\hline
SCAN {\em s} SRC NAME  & string          & systematic name to match that used in {\tt .input file} \\
SCAN {\em s} NUM POLY  & $\ge 1$         & number of polynomials covering scan ({\em nPoly}$_\mathit{s}$) \\
&& Everything below is duplicated {\em nPoly} times. \\
\hline
SCAN {\em s} POLY {\em p} MJD  & integer $\ge 0$ & the start MJD of this polynomial \\
SCAN {\em s} POLY {\em p} SEC  & integer $\ge 0$ & the start sec of this polynomial \\
&& Everything below is duplicated {\em nTelescope} times. \\
ANT {\em a} DELAY (us) & {\em order}+1 numbers & terms of delay polynomial \\
ANT {\em a} DRY (us)   & {\em order}+1 numbers & terms of dry atmosphere \\
ANT {\em a} WET (us)   & {\em order}+1 numbers & terms of wet atmosphere \\
ANT {\em a} U (m)      & {\em order}+1 numbers & terms of baseline $u$ \\ 
ANT {\em a} V (m)      & {\em order}+1 numbers & terms of baseline $v$ \\ 
ANT {\em a} W (m)      & {\em order}+1 numbers & terms of baseline $w$ \\ 
\hline
\end{tabular}
\end{center}






% .input ----------------------------------------------------------------------

\subsection{.input} \label{sec:input}

This section describes the {\tt .input} file format used by {\tt mpifxcorr} to drive correlation.
Because NRAO-DiFX 1.0 uses a non-standard branch of {\tt mpifxcorr} some of the data fields will differ from those used in the official version, either in parameter name or in the available range of values.
Currently the parameters must be in the order listed here.
To get the most out of this section it is advisable to look at an actual file while reading.
An example file is stashed at \url{http://www.aoc.nrao.edu/~wbrisken/NRAO-DiFX-1.1/} .
In the tables below, numbers are assumed to floating point unless otherwise stated.

Note that the input file format has undergone a few minor changes since NRAO-DiFX version 1.0.

\subsubsection{Common settings table}

Below are the keywords and allowed values for entries in the common settings table.
This table begins with header 
\begin{itemize}
\item[] {\tt \verb+# COMMON SETTINGS ##!+} 
\end{itemize}
This is always the first table in a {\tt .input} file.

\begin{center}
\begin{tabular}{l l l}
\hline
Key & Units/allowed values & Comments \\
\hline
DELAY FILENAME     & string & name and full path to {\tt .delay} file \\
UVW FILENAME       & string & name and full path to {\tt .uvw} file \\
CORE CONF FILENAME & string & name and full path to {\tt .threads} file \\
EXECUTE TIME (SEC) & integer seconds & observe time covered by this {\tt .input} file \\
START MJD          & integer MJD & start date \\
START SECONDS      & integer seconds & start time \\
ACTIVE DATASTREAMS & integer $\ge 2$ & number of antennas ({\em nAntenna}) \\
ACTIVE BASELINES   & integer $\ge 1$ & number of baselines to correlate ({\em nBaseline}) \\
VIS BUFFER LENGTH  & integer $\ge 1$ & the number of concurrent integrations to allow \difxoneone \\
OUTPUT FORMAT      & boolean & always {\tt SWIN} here \\
OUTPUT FILENAME    & string & name of output {\tt .difx} directory \\
\hline
\end{tabular}
\end{center}

Typically, $\mathit{nBaseline} = \mathit{nAntenna} \cdot (\mathit{nAntenna}-1)/2$.  Autocorrelations are not included in this count.

\subsubsection{Configurations table}

Below are the keywords and allowed values for entries in the configurations table.
This table begins with header 
\begin{itemize}
\item[] {\tt \verb+# CONFIGURATIONS ###!+}
\end{itemize}
Two indexes are used for repeated keys.  
The index over datastream (antenna) is {\em d}, running from 0 to {\em nAntenna} - 1 and the index over baseline is {\em b}, running from 0 to {\em nBaseline} - 1.

\begin{center}
\begin{tabular}{l l l}
\hline
Key & Units/allowed values & Comments \\
\hline
NUM CONFIGURATIONS & integer $\ge 1$ & number of modes in file ({\em nConfig}) \\
\hline
CONFIG SOURCE      & string & name of configuration \\
INT TIME (SEC)     & seconds & integration time \\
NUM CHANNELS       & integer $\ge 1$ & number of channels (FFT size, {\em nFFT}, is twice this) \\
CHANNELS TO AVERAGE& integer $\ge 1$ & not yet supported (set to 1) \\
OVERSAMPLE FACTOR  & integer $\ge 1$ & total oversampling factor of baseband data  \difxoneone \\
DECIMATION FACTOR  & integer $\ge 1$ & portion of oversampling to handle by decimation \difxoneone \\
BLOCKS PER SEND    & integer $\ge 1$ & number of FFT sizes to send at a time to a core \\
GUARD BLOCKS       & integer $\ge 0$ & number of extra blocks to send for overlap \\
POST-F FRINGE ROT  & boolean & fringe rotate after FFT?  Always {\em FALSE} here \\
QUAD DELAY INTERP  & boolean & use quadratic, not linear, delay interpolation \\
WRITE AUTOCORRS    & boolean & enable auto-correlations; {\em TRUE} here \\
PULSAR BINNING     & boolean & enable pulsar mode; {\em FALSE} for now \\
PULSAR CONFIG FILE & string & ({\em only if BINNING is True}) see \S~\ref{sec:psrconfigfile} \\
DATASTREAM {\em d} INDEX & integer $\ge 0$ & DATASTREAM table index, starting at 0 \\
BASELINE {\em b} INDEX   & integer $\ge 0$ & BASELINE table index, starting at 0 \\
\hline
\end{tabular}
\end{center}

\subsubsection{Frequency table} \label{table:freq}

Below are the keywords and allowed values for entries in the frequency table which defines all possible sub-bands used by the configurations in this file.
Each sub-band of each configuration is mapped to one of these through a value in the datastream table (\S\ref{table:datastream}).
Each entry in this table has three parameters which are replicated for each frequency table entry.
This table begins with header 
\begin{itemize}
\item[] {\tt \verb+# FREQ TABLE #######!+}
\end{itemize}
The table below uses {\em f} to represent the frequency index, which ranges from 0 to {\em nFreq} - 1.

\begin{center}
\begin{tabular}{l l l}
\hline
Key & Units/allowed values & Comments \\
\hline
FREQ ENTRIES & integer $\ge 1$ & number of frequency setups ({\em nFreq}) \\
\hline
FREQ (MHZ) {\em f} & MHz & sky frequency at band edge \\
BW (MHZ) {\em f} & MHz & bandwidth of sub-band \\
SIDEBAND {\em f} & {\tt U} or  {\tt L} & net sideband of sub-band \\
\hline
\end{tabular}
\end{center}

\subsubsection{Telescope table}

Below are the keywords and allowed values for entries in the telescope table which tabulates antenna names and their associated peculiar clock offsets, and the time derivatives of these offsets.
Much of the other antenna-specific information is stored in the datastream table (\S\ref{table:datastream}).
Each datastream of each configuration is mapped to one of these through a value in the datastream table.
Each entry in this table has three parameters which are replicated for each telescope table entry.
This table begins with header 
\begin{itemize}
\item[] {\tt \verb+# TELESCOPE TABLE ##!+}
\end{itemize}
The table below uses {\em a} to represent the antenna index, which ranges from 0 to {\em nAntenna} - 1.

\begin{center}
\begin{tabular}{l l l}
\hline
Key & Units/allowed values & Comments \\
\hline
TELESCOPE ENTRIES & integer $\ge 1$ & number of antennas ({\em nAntenna}) \\
\hline
TELESCOPE NAME {\em a} & string & abbreviation of antenna name \\
CLOCK DELAY (us) {\em a} & $\mu$sec & clock error at start of associated {\tt .delay} file \\
CLOCK RATE(us/s) {\em a} & $\mu$sec/sec & rate at which antenna clock is drifting \\
\hline
\end{tabular}
\end{center}

Note that the reference time for the clock offset is the start of the {\tt .calc} file, not the start of the {\tt .input} file!
This start time is always in integer number of seconds since beginning of day.

\subsubsection{Datastream table} \label{table:datastream}

The datastream table begins with header 
\begin{itemize}
\item[] {\tt \verb+# DATASTREAM TABLE #!+}
\end{itemize}
The table below uses {\em f} to represent the frequency index, which ranges from 0 to {\em nFreq} - 1.
A second index, {\em i}, is used to cover the range 0 to $\mathit{nBB}$ - 1, where the total number of basebands is given by $\mathit{nBB} \equiv \sum_f \mathit{nPol}_f$.
In the DiFX system, all sub-bands must have the same polarization structure, so $\mathit{nBB} = \mathit{nFreq} \cdot \mathit{nPol}$.

\begin{center}
\begin{tabular}{l l l}
\hline
Key & Units/allowed values & Comments \\
\hline
DATASTREAM ENTRIES & integer $\ge 1$ & number of antennas ({\em nDatastream}) \\
DATA BUFFER FACTOR & integer $\ge 1$ &  \\
NUM DATA SEGMENTS  & integer $\ge 1$ &  \\
\hline
TELESCOPE INDEX & integer $\ge 0$ & telescope table index of datastream \\
TSYS & Kelvin & if zero (normal in NRAO usage), don't scale data by {\em tsys} \\
DATA FORMAT & string & data format \\
QUANTISATION BITS & integer $\ge 1$ & bits per sample \\
DATA FRAME SIZE & integer $\ge 1$ & bytes in one frame(or file) of data \difxoneone \\
DATA SOURCE & string & {\tt FILE} (see \S\ref{sec:datafiles}) or {\tt MODULE} for Mark5 playback \difxoneone \\
FILTERBANK USED & boolean & currently only {\tt FALSE} \\
NUM FREQS & integer $\ge 0$ & number of different frequencies for this datastream \\
\hline
FREQ TABLE INDEX {\em f} & integer $\ge 0$ & \\
CLK OFFSET {\em f} (us) & $\mu$sec & \\
NUM POLS {\em f} & 1 or 2 & for NRAO usage, all such parameters must be the same \\
\hline
INPUT BAND {\em i} POL & {\em R} or {\em L} & polarization identity \\
INPUT BAND {\em i} INDEX & integer $\ge 1$ & index to frequency setting array above; {\em nBB} per entry \\
\hline
\end{tabular}
\end{center}

\subsubsection{Baseline table}

In order to retain the highest level of configurability, each baseline can be independently configured at some level.
This datastream table begins with header 
\begin{itemize}
\item[] {\tt \verb+# BASELINE TABLE ###!+}
\end{itemize}
The baseline table has multiple entries, each one corresponding to a pair of antennas, labeled {\tt A} and {\tt B} in the table.
For each of {\em nBaseline} baseline entries, {\em nFreq} sub-bands are processed, and for each a total of {\em nProd} polarization products are formed.
Indexes for each of these dimensions are {\em b}, {\em f} and {\em p} respectively, each starting count at 0.
Within the DiFX context, all baselines must have the same {\em nFreq} and {\em nProd}, though this is not a requirement of {\tt mpifxcorr} in general.

\begin{center}
\begin{tabular}{l l l}
\hline
Key & Units/allowed values & Comments \\
\hline
BASELINE ENTRIES & integer $\ge 1$ & number of entries in table, {\em nBaseline} \\
\hline
D/STREAM A INDEX {\em b} & integer $\ge 0$ & datastream table index of first antenna \\
D/STREAM B INDEX {\em b} & integer $\ge 0$ & datastream table index of second antenna \\
NUM FREQS {\em b} & integer $\ge 1$ & number of frequencies on this baseline, {\em nFreq$_\mathit{b}$} \\
POL PRODUCTS {\em b}/{\em f} & integer $\ge 1$ & number of polarization products, {\em nProd$_\mathit{b}$} \\
D/STREAM A BAND {\em p} & integer $\ge 0$ & index to frequency array in datastream table \\
D/STREAM B BAND {\em p} & integer $\ge 0$ & same as abovem, but for antenna {\tt B}, not {\tt A} \\
\hline
\end{tabular}
\end{center}

\subsubsection{Data Table}

In the following table, {\em d} is the datastream index, ranging from 0 to {\em nDatastream} - 1 and {\em f} is the file index ranging from 0 to {\em nFile$_\mathit{d}$}.

\begin{center}
\begin{tabular}{l l l}
\hline
Key & Units/allowed values & Comments \\
\hline
D/STREAM {\em d} FILES & integer $\ge 1$ & number of files {\em nFile$_\mathit{d}$} associated with datastream {\em d} \\
FILE {\em d}/{\em f} & string & name of file or module associated with datastream {\em d} \\
\hline
\end{tabular}
\end{center}

For datastreams reading off Mark5 modules, {\em nFile} will always be 1 and the filename is the {\em VSN} of the module being read.







% .joblist --------------------------------------------------------------------

\subsection{.joblist} \label{sec:joblistfile}

\vspace{-20pt}\hspace{80pt}
\difxonefive

\vspace{7pt}

\noindent
A single {\tt .joblist} file is written by {\tt vex2difx} (\S\ref{sec:vex2difx}) as it produces the DiFX {\tt .input} (and other) files for a given correlator pass.
This file contains the list of jobs to run and some versioning information that allows improved accountability of the software versions being used.
This file us used by {\tt difxqueue} and {\tt makefits} to ensure that a complete set of jobs is accounted for.
The file is composed of two parts: a header line and one line for each job.
The header line consists of a series of {\em key=value} pairs.  
Each {\em key} and {\em value} must have no whitespace and no whitespace should separate these words from their connecting {\tt =} sign.
While any number of key-value pairs may be specified, the following ones (which are case sensitive) are expected to be present:
\begin{enumerate}
\item {\tt exper} : the name of the experiment, including the segment code
\item {\tt v2d} : the {\tt vex2difx} input file used to produce the jobs of this pass
\item {\tt pass} : the name of the correlator pass
\item {\tt mjd} : the modified Julian day when {\tt vex2difx} created this file
\item {\tt DiFX} : the version name for the DiFX deployment (the value of {\tt \$DIFX\_VERSION} when {\tt vex2difx} was run)
\item {\tt vex2difx} : the version of {\tt vex2difx} that was run
\end{enumerate}
Each additional line contains information for one job in the pass.
The columns are:
\begin{enumerate}
\item {\em jobName} : the name/prefix of the job
\item {\em mjdStart} : the observe start time of the job
\item {\em mjdStop} : the observe stop time of the job
\item {\em nAnt} : the number of antennas in the job
\item {\em tOps} : estimated number of trillion floating point operations required to run the job
\item {\em outSize} : estimated FITS file output size (MB)
\end{enumerate}
Usually the comment character {\tt \#} followed by a list of station codes is appended to the end of each line.








% .jobmatrix ------------------------------------------------------------------

\subsection{.jobmatrix} \label{sec:jobmatrixfile}

\vspace{-20pt}\hspace{105pt}
\difxonefive

\vspace{7pt} 

\noindent
As of version~2.6 of {\tt difx2fits} a file with extension {\tt .jobmatrix} will be written for each {\tt .FITS} file created.
This file is meant as a summary for human use and as such does not have a format that should be considered fixed.
The file contains a 2-dimensional map (antenna number vs.\ time) of which jobs contributed to the {\tt .FITS} file.





% .log ------------------------------------------------------------------------

\subsection{.log} \label{sec:log}

\vspace{-20pt}\hspace{65pt}
\difxoneone

\vspace{7pt}

\noindent
When generation of sniffer output files is not disabled, each {\tt .FITS} file written by {\tt difx2fits} will be accompanied by a corresponding {\tt .log} file. 
This file contains a summary of the contents of that {\tt .FITS} file.
It is analogous to the {\tt logfile.lis} file produced by the old {\tt FITSsniffer} program.
This file is free-form ASCII that is intended for viewing by human eyes, and is should not be used as input to any software as the format is not guaranteed to remain constant.







% .machines -------------------------------------------------------------------

\subsection{.machines} \label{sec:machines}

The {\tt .machines} file is used by {\tt mpirun} to determine which machines will run {\tt mpifxcorr}.
This is a text file containing a list of computers, one to a line possibly with additional options listed, on which to spawn the software correlator process.
As a general rule the MPI rank, a unique number for each process that starts at 0, are allocated in the order that the computer names are listed.
This general rule can break down in cases where the same computer name is listed more than once; the behavior in this case depends on the MPI implementation being used.
MPI rank 0 will always be the manager process.
Ranks 1 through {\em nDatastream} will each be a datastream process.
Additional processes will be computing (core) processes.
If more processes are specified for {\tt mpirun} with the {\tt -np} option than there are lines in this file, the file will be read again from the top, so the processes will be assigned in a cyclic fashion (again, this depends somewhat on the MPI implementation and the other parameters passed to {\tt mpirun}; for DiFX with OpenMPI, this assumes {\tt --bynode} is used).
If the program {\tt startdifx} is used to start the correlation process, the number of processes to start is determined by the number of lines in this file.
If wrapping to the top of this file is desired, dummy comment lines (beginning with \#) can be put at the end of the {\tt .machines} file to artificially raise the number of processes to spawn.
Within DiFX, this file is typically produced by {\tt genmachines}.
Keep in mind that this file is directly read by the MPI execution program {\tt mpirun} and the format of the file may differ depending on the MPI implementation that you are using.
With OpenMPI appending {\tt slots=1 max-slots=1} to the end of each line ensures that a single instance of {\tt mpifxcorr} is run on that machine.
If both a datastream process and a core process are to be run on the same computer, then using options {\tt slots=1 max-slots=2} might be appropriate.









% .oms ------------------------------------------------------------------------

\subsection{.oms} \label{sec:oms}

A {\tt .oms} file is written by {\tt sched} and contains machine (and human) readable information that is useful in setting up correlator jobs.
In the case of the VLBA DiFX correlator, program {\tt oms2v2d} (\S\ref{sec:oms2v2d}) uses this file to prepare a template {\tt .v2d} file (\S\ref{sec:v2d}) that contains some information not available in the vex file, such as intended integration time and number of channels.








% .params ---------------------------------------------------------------------

\subsection{.params} \label{sec:params}
\vspace{-20pt}\hspace{85pt}
\difxonefive

\vspace{7pt}

\noindent
A file with extension {\tt .params} is written by {\tt vex2difx} (\S\ref{sec:vex2difx}) when it is provided with the {\tt -o} option.
This file is a duplicate of the {\tt .v2d} file that was supplied but with all unspecified parameters listed with the defaults that they assumed.
The format is exactly the same as the {\tt .v2d} files; see \S\ref{sec:v2d} for documentation of the format.
The {\tt .params} file can be used as a legal {\tt .v2d} file if necessary.







% pcal ------------------------------------------------------------------------

\subsection{pcal} \label{sec:pcal}

A file called {\tt pcal} is created when program {\tt vlog} operates on the {\tt cal.vlba} file.
This file contains three measurements: the cable length calibration, pulse calibration, and state counts.
This file contains two kinds of lines.
Comment lines begin with an octothorpe (\#) and contain no vital information.
Data lines always contain 8 fixed-size fields:
\begin{enumerate}
\item {\em antId} : Station name abbreviation, e.g., {\tt LA}
\item {\em day} : Time centroid of measurement (day of year, including fractional portion)
\item {\em dur} : Duration of measurement (days)
\item {\em cableCal} : Cable calibration measurement (picoseconds)
\item {\em nPol} : Number of polarizations with measurements 
\item {\em nBand} : Number of sub-bands with measurements 
\item {\em nTone} : Number of pulse cal tones detected per band per polarization, possibly zero 
\item {\em nState} : Number of state count states measured per band per polarization, possibly zero
\item {\em nRecChan} : Number of record channels at time of measurement ($\le$ \em{nPol * nBand})
\end{enumerate}
Following these eight fields are two variable-length arrays of numbers.
The first variable-length field is the pulse cal data field consisting of {\em nPol*nBand*nTone} groups of four numbers.
The first member of this group is the recorder channel number (zero-based) corresponding to the measurement.
The second member of this group is the tone sky frequency (MHz).
The third and fourth are respectively the real and imaginary parts of the tone measured at the given sky frequency.
The order in which the groups are presented (in `C' language array syntax, as used throughout this document) is $[${\em nPol}$][${\em nBand}$][${\em nTone}$]$.
Note that if there are fewer than {\em nPol*nBand} record channels, the record channel will be $-1$ for some groups.
The second variable-length field is the state count data.
For each band of each polarization, {\em nState} + 1 values are listed.
The first number is the record channel number or -1 if that polarization/band combination was not observed or monitored.
The remainder contain state counts.
{\em nState} can be either 0 or $2^{\mathit nBit}$, where {\em nBit} is the number of quantization bits.
The order in which these groups are listed is $[${\em nPol}$][${\em nBand}$]$.







% .polyco ---------------------------------------------------------------------

\subsection{.polyco} \label{sec:polycofile} 

\vspace{-20pt}\hspace{85pt}
\difxoneone

\vspace{7pt}

\noindent
A polyco file contains a single polynomial for pulse phase that is valid for a fraction (up to 100\%) of a job file.
An additional numeric suffix is appended to the filename specifying the polynomial index for a particular {\tt .pulsar} file that shares the same base name.
The format of the file is the same as a Tempo pulsar file \cite{tempo}.







% .pulsar ------------------------------------------------- REVISE ME ---------

\subsection{.pulsar} \label{sec:psrconfigfile} 

\vspace{-20pt}\hspace{85pt}
\difxoneone

\vspace{7pt}

\noindent
Within the scope of DiFX, all pulsar configuration files will adopt the {\tt .pulsar} suffix.
There will be a separate .pulsar file for each pulsar-configuration combination in each {\\ .fx} file that is converted to DiFX input file format by {\tt job2difx}; a series of 4 numbers preceding the suffix indicates which source number, frequency id, correlator table entry, and configuration the file belongs to.
Multiple subjobs derived from a single {\tt .fx} file may share {\tt .pulsar} files.

\begin{center}
\begin{tabular}{l l l}
\hline
Key & Units/allowed values & Comments \\
\hline
NUM POLYCO FILES       & integer $\ge 1$ & number of polyco files to look at ({\em nPoly}) \\
&& The row below is duplicated {\em nPoly} times. \\
POLYCO FILE {\em p}    & string & filename of {\em p}$^{\mathrm{th}}$ polyco file; see \S~\ref{sec:polycofile} \\
NUM PULSAR BINS        & integer $\ge 1$ & number of pulsar bins to consider ({\em nBin}) \\
SCRUNCH OUTPUT         & boolean         & TRUE for regular ``gating''; FALSE not yet supported \\
&& The two rows below are duplicated {\em nBin} times. \\
BIN PHASE END {\em b}  & float & pulse phase (in periods) of end of bin {\em b} \\
BIN WEIGHT {\em b}     & float $\ge 0$ & weight to apply to bin {\em b} \\
\hline
\end{tabular}
\end{center}








% .rate -----------------------------------------------------------------------

\subsection{.rate} \label{sec:rate}

In order to deliver model information to {\tt difx2fits} that is not needed by {\tt mpifxcorr}, an additional file is written by {\tt calcif2}.
The format of this file is most similar to that of the {\tt .uvw} files, however the payload in this file is very different.
Note that the values of the delays in this file have the opposite sign as compared to those generated by CALC and those stored in {\tt .FITS} files.

The file consists entirely of key-value pairs separated by a colon.
There are three sections in the {\tt .rate} file; these sections are not separated by any explicit mark in the file.

The first section contains values that are fixed for the entire experiment and at all antennas --- all data in this section is scalar.
In the following table, all numbers are assumed to be floating point unless further restricted.
The keys and allowed values in this section are summarized below.
Keys preceded by $\star$ are optional.

\begin{center}
\begin{tabular}{l l l}
\hline
Key & Units/allowed values & Comments \\
\hline
\Oa{CALC SERVER}        & string         & name of the calc server computer used \\
\Oa{CALC PROGRAM}       & integer        & RPC program ID of the calc server used \\
\Oa{CALC VERSION}       & integer        & RPC version ID of the calc server used \\
START MJD          & MJD + fraction & start time of this subjob \\
START YEAR         & integer        & calendar year of START MJD \\
START MONTH        & integer        & calendar month of START MJD \\
START DAY          & integer        & day of calendar month of START MJD \\
START HOUR         & integer        & hour of START MJD \\
START MINUTE       & integer        & minute of START MJD \\
START SECOND       & integer        & second of START MJD \\
INCREMENT (SECS)   & integer        & seconds between computed model points ({\em inc}) \\
\hline
\end{tabular}
\end{center}

The second section contains antenna(telescope) specific information.
This section is absolutely identical to the telescope section describe for the {\tt .calc} file (see second table in \S\ref{sec:calc}) so its description will be omitted here.

The third and final section contains the scan-based information.
First is a line indicating the number of scans to follow.
Then for each scan, numbered by {\em s} ranging from 0 to {\em nScan} - 1, there are 5 lines containing information about the scan, including the number of sampled points within that scan, {\em nPoint}.
Finally there are {\em nPoint}$_\mathit{s}$ + 3 lines containing the tabulated data, numbered $-$1 through {\em nPoints}$_\mathit{s}$ + 1, indexed with {\em p}.
Note that this includes one sample before the start of the scan and at least one after the scan end, allowing for a quadratic interpolation across the entire scan.
This information is summarized in the following table:

\begin{center}
\begin{tabular}{l l l}
\hline
Key & Units/allowed values & Comments \\
\hline
NUM SCANS              & integer $\ge 1$ & number of scans ({\em nScan}). \\
&& The rows below are duplicated {\em nScan} times. \\
\hline
SCAN {\em s} POINTS    & $\ge 1$         & duration of scan in units of {\em inc} ({\em nPoints}$_\mathit{s}$) \\
SCAN {\em s} START PT  & integer $\ge 0$ & start time of scan in units of {\em inc} since MJD START \\
SCAN {\em s} SRC NAME  & string          & systematic name to match that used in {\tt .input file} \\
SCAN {\em s} SRC RA    & radians         & J2000 right ascension\\
SCAN {\em s} SRC DEC   & radians         & J2000 declination \\
\hline
RELATIVE INC {\em p}   & array; see below & {\em nPoint}$_\mathit{s}$ + 3 of these lines per scan \\
\hline
\end{tabular}
\end{center}

Like for the {\tt .delay} and {\tt .uvw} files, the values reported in this file extend one full mode increment ({\em inc}) before and after the actual duration of the scan, and hence will overlap in time by 2$\times$ {\em inc} with consecutive scans.
The {\tt RELATIVE INC} lines contain triplets of values for each antenna.
The first member of the triplet is the time derivative of the geometric delay, in $\mu$s/s.
The second and third elements are the estimated dry and wet atmospheric delay terms, respectively, in $\mu$s.

This file is typically produced by {\tt calcif2}.







% .shelf ----------------------------------------------------------------------

\subsection{.shelf} \label{sec:shelf}

\vspace{-20pt}\hspace{75pt}
\difxonefive

\vspace{7pt}

\noindent
The vex file format (see \S\ref{sec:dotvex} and references within) does not have a formal slot to record the shelf location of media, so {\tt db2vex} stashes the shelf location in a separate file.
This information is critical for the correlator operators to know where to find modules for a project and for analysts preparing correlator jobs to know if media have arrived.
The {\tt .shelf} file is used by {\tt vex2difx} when writing {\tt .calc files}.
It can also be used as input to {\tt getshelf}.
The file format is very simple.
One row is used for each module that was used in the observation.
Typically rows are sorted in the same order as antennas in the {\tt .input} file.
The comment character is {\tt \#} -- any text following this character on a line is ignored.
Each line contains 3 white-space separated columns:
\begin{enumerate}
\item {\em antId} : The typically 2 letter station abbreviation
\item {\em vsn} : The volume serial number of the media (e.g., {\tt NRAO-123})
\item {\em shelf} : The shelf location, which can be any string without whitespace (e.g., {\tt BD89}), or {\tt none} if the media is not at the correlator 
\end{enumerate}







% .threads --------------------------------------------------------------------

\subsection{.threads} \label{sec:threads}

The {\tt .threads} file tells {\tt mpifxcorr} how many threads to start on each processing node.
Within DiFX, this file is typically produced by {\tt genmachines}.
The {\tt.threads} file has a very simple format.
The first line starts with {\tt NUMBER OF CORES:}.
Starting at column 21 is an integer that should be equal to the number of processing nodes ({\em nCore}) specified in the corresponding {\tt .machines} file.
Each line thereafter should contain a single integer starting at column 1.
There should be {\em nCore} such lines.








% tsys ------------------------------------------------------------------------

\subsection{tsys} \label{sec:tsys}

A file called {\tt tsys} is created when program {\tt vlog} operates on the {\tt cal.vlba} file.
This file contains measurements of the system temperature and name of receiver for each baseband channel.
This file contains two finds of lines.
Comment lines begin with an octothorpe (\#) and contain no vital information.
Data lines always contain 4 fixed-size fields:
\begin{enumerate}
\item {\em antId} : Station name abbreviation, e.g., {\tt LA}
\item {\em day} : Time centroid of measurement (day of year, including fractional portion)
\item {\em dur} : Duration of measurement (days); currently set to zero for lack of information
\item {\em nRecChan} : Number of baseband channels recorded 
\end{enumerate}
Following these 4 fields are {\em nRecChan} pairs of values, one for each baseband channel.
The first element of each pair is the system temperature (in K) and the second is the receiver name (e.g., {\tt 4cm}, or {\tt 7mm}).







% weather ---------------------------------------------------------------------

\subsection{weather} \label{sec:weather}

A file called {\tt weather} is created when program {\tt vlog} operates on the {\tt cal.vlba} file.
This file contains tabulated values of various meteorological measurements.
This file contains two finds of lines.
Comment lines begin with an octothorpe (\#) and contain no vital information.
Data lines always contain 8 fixed-size fields:
\begin{enumerate}
\item {\em antId} : Station name abbreviation, e.g., {\tt LA}
\item {\em day} : Time of measurement (day of year, including fractional portion)
\item {\em T} : Ambient temperature (Centigrade)
\item {\em P} : Pressure (mbar)
\item {\em dewPoint} : Dew point (Centigrade)
\item {\em windSpeed} : Wind speed (m/s)
\item {\em windDir} : Wind direction (degrees E of N)
\item {\em precip} : Accumulated rain since UT midnight (cm)
\item {\em windGust} : Maximum wind gust over collection period (m/s)
\end{enumerate}








% .wts ------------------------------------------------------------------------

\subsection{.wts} \label{sec:wts}

\vspace{-20pt}\hspace{60pt}
\difxoneone

\vspace{7pt}

\noindent
When generation of sniffer output files is not disabled, each {\tt .FITS} file written by {\tt difx2fits} will be accompanied by a corresponding {\tt .wts} file. 
This file contains statistics of the data weights, typically dominated by the completeness of records as determined by the data transport system, over a typically 30 second long period.

The first line is simply the observe code, e.g., {\tt MT831} .

Each additional line in the file is a complete record for a given antenna for a given interval, containing information for each baseband channel separately.
The format of these lines is as follows:

\begin{center}
\begin{tabular}{l l l}
\hline
Key & Units/allowed values & Comments \\
\hline
{\it MJD}                & integer $\ge 1$     & MJD day number corresponding to line \\
{\it hour}               & $\ge 0.0$, $< 24.0$ & hour within day \\
{\it antenna number}     & $\ge 1$             & antenna table index \\
{\it antenna name}       & string              & \\
{\it n}$_{\mathrm{BBC}}$ & $\ge 1$             & Number of baseband channels \\
{\it mean weight}        & $\ge 0.0$           & This column repeated $n_{\mathrm{BBC}}$ times \\
{\it min weight}         & $\ge 0.0$           & This column repeated $n_{\mathrm{BBC}}$ times \\
{\it max weight}         & $\ge 0.0$           & This column repeated $n_{\mathrm{BBC}}$ times \\
\hline
\end{tabular}
\end{center}

\noindent
This file can be used directly with plotting program {\tt plotwt} or used more automatically with {\tt difxsniff}.







% .uvw ------------------------------------------------------------------------

\subsection{.uvw} \label{sec:uvw}

The {\tt .uvw} file contains tabulated baseline vectors.
The values in this file are interpolated to produce values valid for integration centroids and included in the visibility file output by {\tt mpifxcorr}.
This file consists entirely of key-value pairs separated by a colon.
There are three sections in the {\tt .uvw} file; these sections are not separated by any explicit mark in the file.

The first section contains values that are fixed for the entire experiment and at all antennas --- all data in this section is scalar.
In the following table, all numbers are assumed to be floating point unless further restricted.
The keys and allowed values in this section are summarized below:

\begin{center}
\begin{tabular}{l l l}
\hline
Key & Units/allowed values & Comments \\
\hline
START YEAR         & integer        & calendar year of START MJD \\
START MONTH        & integer        & calendar month of START MJD \\
START DAY          & integer        & day of calendar month of START MJD \\
START HOUR         & integer        & hour of START MJD \\
START MINUTE       & integer        & minute of START MJD \\
START SECOND       & integer        & second of START MJD \\
INCREMENT (SECS)   & integer        & seconds between tabulated baseline vectors ({\em inc}) \\
\hline
\end{tabular}
\end{center}

The second section contains antenna(telescope) specific information with the following format:
\begin{center}
\begin{tabular}{l l l}
\hline
Key & Units/allowed values & Comments \\
\hline
NUM TELESCOPES               & integer $\ge 1$& number of telescopes ({\em nTelescope}). \\
&& The rows below are duplicated {\em nTelescope} times. \\
\hline
TELESCOPE {\em t} NAME       & string & upper case antenna name abbreviation \\
TELESCOPE {\em t} MOUNT      & string & the mount type: altz, equa, xyew, or xyns \\
TELESCOPE {\em t} X (m)      & meters & X geocentric coordinate of antenna at date \\
TELESCOPE {\em t} Y (m)      & meters & Y geocentric coordinate of antenna at date \\
TELESCOPE {\em t} Z (m)      & meters & Z geocentric coordinate of antenna at date \\
\hline
\end{tabular}
\end{center}

The third and final section contains the scan-based information.
First is a line indicating the number of scans to follow.
Then for each scan, numbered by {\em s} ranging from 0 to {\em nScan} - 1, there are 5 lines containing information about the scan, including the number of sampled points within that scan, {\em nPoint}.
Finally there are {\em nPoint}$_\mathit{s}$ + 3 lines containing the tabulated data, numbered $-$1 through {\em nPoints}$_\mathit{s}$ + 1, indexed with {\em p}.
Note that this includes one sample before the start of the scan and at least one after the scan end, allowing for a quadratic interpolation across the entire scan.
This information is summarized in the following table:

\begin{center}
\begin{tabular}{l l l}
\hline
Key & Units/allowed values & Comments \\
\hline
NUM SCANS              & integer $\ge 1$ & number of scans ({\em nScan}). \\
&& The rows below are duplicated {\em nScan} times. \\
\hline
SCAN {\em s} POINTS    & $\ge 1$         & duration of scan in units of {\em inc} ({\em nPoints}$_\mathit{s}$) \\
SCAN {\em s} START PT  & integer $\ge 0$ & start time of scan in units of {\em inc} since MJD START \\
SCAN {\em s} SRC NAME  & string          & systematic name to match that used in {\tt .input file} \\
SCAN {\em s} SRC RA    & radians         & J2000 right ascension\\
SCAN {\em s} SRC DEC   & radians         & J2000 declination \\
\hline
RELATIVE INC {\em p}   & array; see below & {\em nPoint}$_\mathit{s}$ + 3 of these lines per scan \\
\hline
\end{tabular}
\end{center}

Like for the {\tt .delay} and {\tt .rate} files, the values reported in this file extend one full mode increment ({\em inc}) before and after the actual duration of the scan, and hence will overlap in time by 2$\times$ {\em inc} with consecutive scans.
The {\tt RELATIVE INC} lines contain triplets of values for each antenna, representing in order the $u$, $v$, and $w$ components of the baseline vector, all expressed in meters from the earth center.

This file is typically produced by {\tt calcif2}.








% .v2d ------------------------------------------------------------------------

\subsection{.v2d} \label{sec:v2d}

\vspace{-20pt}\hspace{65pt}
\difxonefive

\vspace{7pt}

\noindent
The {\tt .v2d} file is used to specify correlation options to {\tt vex2difx} and adjust the way in which it forms DiFX input files based on the {\tt .vex} file.
The {\tt .v2d} file consists of a number of global parameters that affect the way that jobs are created and several sections that can customize correlation on a per-source, per mode, or per scan basis.  
All parameters (those that are global and those that reside inside sections) are specified by a parameter name, the equal sign, and one value, or a comma-separated list of values, that cannot contain whitespace.  
Whitespace is not required except to keep parameter names, values, and section names separate.  
All parameter names and values are case sensitive except for source names and antenna names.  
The \# is a comment character; any text after this on a line is ignored.

Most parameters are one of the following types:
\begin{itemize}
\item {\bf bool} : A boolean value that can be True or False.  Any value starting with {\tt 0}, {\tt f}, {\tt F}, or {\tt -} will be considered False and otherwise True.
\item {\bf float} : A floating point number.  Can be of the forms: {\tt 1.23}, {\tt 1.2e-4}, {\tt -12.6}, {\tt 4}
\item {\bf int} : An integer.
\item {\bf string} : Any sequence of printable(non-whitespace) characters.  Certain fields require strings of a maximum length or certain form.
\item {\bf date} : A date field; see below.
\item {\bf array} :  Array can be of any of the four above types and are indicated by enclosing brackets, e.g., $[$int$]$.  The empty list is indicated with $[ ]$ which is usually implied to be all-inclusive.
\end{itemize}

All times used in {\tt vex2difx} are in Universal Time and are internally represented as a double precision value.
The integer part of this value is the date corresponding to 0$^h$ UT.
The fractional part, when multiplied by 86400, gives the number of seconds since 0$^h$ UT.
Note that this format does not allow one to specify the actual leap second if one occurs on that day.
Several date formats are supported:
\begin{itemize}
\item {\bf Modified Julian Day} : A decimal MJD possibly including fractional day.  E.g.: {\tt 54345.341944}
\item {\bf Vex time format} : A string of the form: {\tt 2009y245d08h12m24s}
\item {\bf VLBA-like format} : A string of the form: {\tt 2009SEP02-08:12:24}
\item {\bf ISO 8601 format} : A string of the form: {\tt 2009-09-02T08:12:24}
\end{itemize}

Global parameters can be specified one or many per line such as:

\noindent
{\tt maxGap = 2000 \# seconds}

\noindent
or

\noindent
{\tt mjdStart = 52342.522 mjdStop=52342.532}

The following parameter names are recognized:

\begin{center}
\begin{tabular}{l l l l l}
\hline
Name & Type & Units & Defaults & Comments \\
\hline
vex              & string &       &              & filename of the vex file to process; {\bf this is required} \\
mjdStart         & date   &       & obs.\ start  & discard any scans or partial scans before this time \\
mjdStop          & date   &       & obs.\ stop   & discard any scans or partial scans after this time \\
break            & date   &       &              & list of MJD date/times where jobs are forced to be broken \\
minSubarray      & int    &       & 2            & don't make jobs for subarrays with fewer antennas than this \\
maxGap           & float  & sec   & 180          & split an observation into multiple jobs if there are \\
                 &        &       &              & correlation gaps longer than this number \\
singleScan       & bool   &       & False        & if True, split each scan into its own job \\
singleSetup      & bool   &       & True         & if True, allow only one setup per job; True is required \\
                 &        &       &              & for FITS-IDI conversion \\
maxLength        & float  & sec   & 7200         & don't allow individual jobs longer than this amount of time \\
minLength        & float  & sec   & 2            & don't allow individual jobs shorter than this amount of time \\
maxSize          & float  & MB    & 2000         & max FITS-IDI file size to allow \\
dataBufferFactor & int    &       & 32           & the {\tt mpifxcorr} DATABUFFERFACTOR parameter \\
nDataSegments    & int    &       & 8            & the {\tt mpifxcorr} NUMDATASEGMENTS parameter \\
jobSeries        & string &       &              & the base filename of {\tt .input} and {\tt .calc} files to be \\
                 &        &       &              & created; defaults to the base name of the {\tt .v2d} file \\
startSeries      & int    &       & 20           & the default starting number for jobs created \\
sendLength       & float  & sec   & 0.262144     & roughly the amount of data to send at a time from \\
                 &        &       &              & datastream processes to core processes \\
antennas         & $[$string$]$ &  & $[ ]$ = all & a comma separated list of antennas to include in correlation \\
baselines        & $[$string$]$ &  & $[ ]$ = all & a comma separated list of baselines to correlate; see below \\
padScans         & bool   &       & True         & insert non-correlation scans in recording gaps to prevent \\
                 &        &       &              & {\tt mpifxcorr} from complaining \\
invalidMask      & int    &       & 0xFFFF       & this bit-field selects which flag conditions are considered \\
                 &        &       &              & when writing flag file: 1=Recording, 2=On source, 4=Job \\
		 &        &       &              & time range, 8=Antenna in job \\
visBufferLength  & int    &       & 32           & number of visibility buffers to allocate in mpifxcorr \\
simFXCORR        & bool   &       & False        & simulate the VLBA HW correlator $t_{\mathrm{int}}$ and start times \\
overSamp         & int    &       &              & force all basebands to use the given oversample factor \\
mode             & string &       & normal       & mode of operation; see below \\
\hline
\hline
\end{tabular}
\end{center}

The {\em baselines} parameter supports the wildcard character {\tt *} an individual antenna name, or lists of antenna names separated by {\tt +} on each side of a hyphen ({\tt -}).
Multiple baseline designators can be listed.
Examples:
\begin{itemize}
\item {\tt A1-A2} : Only correlate one baseline
\item {\tt A1-A2, A3-A4} : Correlate 2 baselines
\item {\tt *-*} : Correlate all baselines
\item {\tt A1-*} {\em or} {\tt *-A1} : Correlate all baselines to antenna A1
\item {\tt A1+A2-*} : Correlate all baselines to antenna A1 or A2
\item {\tt A1+A2-A3+A4+A5} : Correlate 6 baselines
\end{itemize}

A source section can be used to change the properties of an individual source, such as its position or name.
In the future this is where multiple correlation centers for a given source will be specified.
A source section is enclosed in a pair of curly braces after the keyword SOURCE followed by the name of a source, for example

\begin{Verbatim}[commandchars=\|\[\]]
  SOURCE 3C273
  {
    |bfit[source parameters go here]
  }
\end{Verbatim}

\noindent
or equivalently

\begin{Verbatim}[commandchars=\|\[\]]
  SOURCE 3c273 { |bfit[source parameters go here] }
\end{Verbatim}

\begin{center}
\begin{tabular}{l l l l l}
\hline
Name & Type & Units & Defaults & Comments \\
\hline
ra             &        & J2000 &         & right ascension, e.g., {\tt 12h34m12.6s} or {\tt 12:34:12.6} \\
dec            &        & J2000 &         & declination, e.g., {\tt 34d12'23.1}" or {\tt 34:12:23.1} \\
name           & string &       &         & new name for source \\
calCode        & char   &       & ' '     & calibration code, typically {\tt A}, {\tt B}, {\tt C} for calibrators, {\tt G} for a gated \\
               &        &       &         & pulsar, or blank for normal target \\
\hline
\hline
\end{tabular}
\end{center}


An antenna section allows properties of an individual antenna, such as position, name, or clock/LO offsets, to be adjusted. 
Note that the ``late'' convention is used in {\em clockOffset} and {\em clockRate}, unlike the ``early'' convention used in the {\tt .vex} file itself (see \S\ref{sec:clockconventions}).

\begin{center}
\begin{tabular}{l l l l l}
\hline
Name & Type & Units & Defaults & Comments \\
\hline
name           & string &       &           & new name to assign to this antenna \\
polSwap        & bool   &       & False     & swap the polarizations (i.e., {\tt L} $\Leftrightarrow$ {\tt R}) for this antenna \\
clockOffset    & float  & us    & vex value & overrides the clock offset value from the vex file \\
clockRate      & float  & us/s  & vex value & overrides the clock offset rate value from the vex file \\
clockEpoch     & date   &       & vex value & overrides the epoch of the clock rate value; must be present if \\
               &        &       &           & clockRate or clockOffset parameter is set \\
deltaClock     & float  & us    & 0.0       & adds to the clock offset (either the vex value or the clockOffset above  \\
deltaClockRate & float  & us/s  & 0.0       & adds to the clock rate (either the vex value or the clockRate above \\
X              & float  & m     & vex value & change the X coordinate of the antenna location \\
Y              & float  & m     & vex value & change the Y coordinate of the antenna location \\
Z              & float  & m     & vex value & change the Z coordinate of the antenna location \\
format         & string &       &           & force format to be one of VLBA, MKIV, Mark5B or S2 \\
file           & strings &      & (none)    & a comma separated list of data files to correlate \\
filelist       & string  &      &           & a filename listing files for the DATA TABLE \\
\hline
\hline
\end{tabular}
\end{center}

Setup sections are enclosed in braces after the word SETUP and a name given to this setup section.
The setup name is referenced by a RULE section (see below).
A setup with the special name {\tt default} will be applied to any scans not otherwise assigned to setups by rule sections.
If no setup sections are defined, a setup called {\tt default}, with all default parameters, will be implicitly created and applied to all scans.
The order of setup sections is immaterial.

\begin{center}
\begin{tabular}{l l l l l}
\hline
Name & Type & Units & Defaults & Comments \\
\hline
tInt           & float  & sec   & 2       & integration time \\
nChan          & int    &       & 16      & number of channels per spectral window; must be $2^n$ \\
doPolar        & bool   &       & True    & correlate cross hands when possible \\
blocksPerSend  & int    &       &         & the {\tt mpifxcorr} BLOCKSPERSEND parameter; defaults to a \\
               &        &       &         & value that depends on other parameters \\
specAvg        & int    &       & 1       & how many channels to average together after correlation \\
startChan      & int    &       & 0       & first (unaveraged) channel to include in output \\
nOutChan       & int    &       & {\em nChan} & truncate after this many output channels \\
postFFringe    & bool   &       & False   & do fringe rotation after FFT? \\
binConfig      & string &       & none    & if specified, apply this pulsar bin config file to this setup \\
freqId         & $[$int$]$ &    & $[ ]$ = all& frequency bands to correlate \\
\hline
\hline
\end{tabular}
\end{center}

Earth Orientation Parameter (EOP) data can be provided via one or more EOP sections.
EOP data should be provided either in the {\tt .v2d} file or in the vex file, but not both.
Normally the vex file would be used to set EOP values, but there may be cases (eVLBI?) that want to use the vex file from {\tt sched} without any modification.
Like ANTENNA and SOURCE sections, each EOP section has a name.
The name must be in a form that can be converted directly to a date (see above for legal date formats).
Conventional use suggests that these dates should correspond to 0 hours UT; deviation from this practice is at the users risk.
There are four parameters that should all be set within an EOP section:

\begin{center}
\begin{tabular}{l l l l l}
\hline
Name & Type & Units & Defaults & Comments \\
\hline
tai\_utc & float & sec    & & TAI minus UTC; the leap-second count  \\
ut1\_utc & float & sec    & & UT1 minus UTC; Earth rotation phase \\
xPole    & float & arcsec & & X component of spin axis offset \\
yPole    & float & arcsec & & Y component of spin axis offset \\
\hline
\hline
\end{tabular}
\end{center}

A rule section is used to assign a setup to a particular source name, calibration code (currently not supported), scan name, or vex mode.
The order of rule sections {\em does} matter as the order determines the priority of the rules.
The first rule that matches a scan is applied to that scan.
The correlator setup used for scans that match a rule is determined by the parameter called ``setup''.
A special setup name {\tt SKIP} causes matching scans not to be correlated.
Any parameters not specified are interpreted as fully inclusive.
Note that multiple rule sections can reference the same setup section.
Multiple values may be applied to any of the parameters except for ``setup''.
This is accomplished by comma separation of the values in a single assignment or with repeated assignments.
Thus

\begin{verbatim}
  RUlE rule1
  {
    source = 3C84,3C273
    setup = BrightSourceSetup
  }
\end{verbatim}

\noindent
is equivalent to

\begin{verbatim}
  RULE rule2
  {
    source = 3C84 3C273
    setup = BrightSourceSetup
  }
\end{verbatim}

\noindent
is equivalent to

\begin{verbatim}
  RULE rule3
  {
    source = 3C84
    source = 3C273
    setup = BrightSourceSetup
  }
\end{verbatim}

The names given to rules (e.g., rule1, rule2 and rule3 above) are not used anywhere (yet) but are required to be unique.

\begin{center}
\begin{tabular}{l l l l l}
\hline
Name & Type & Units & Comments \\
\hline
scan           & $[$string$]$ &   & one or more scan name, as specified in the vex file, to select with this rule \\
source         & $[$string$]$ &   & one or more source name, as specified in the vex file, to select with this rule \\
calCode        & $[$char$]$   &   & one or more calibration code to select with this rule \\
mode           & $[$string$]$ &   & one or more modes as defined in the vex file to select with this rule \\
setup          & string       &   & The name of the SETUP section to use, or SKIP if this rule describes scans \\
               &              &   & not to correlate \\
\hline
\hline
\end{tabular}
\end{center}

Note that source names and calibration codes reassigned by source sections are not used.
Only the names and calibration codes in the vex file are compared.

There are currently two modes of operation supported by {\tt vex2difx}.  
The mode used in the vast majority of situations is called {\tt normal} and is the default if none is specified.
Currently one alternative mode, {\tt profile}, is supported.
This mode is useful for generating pulse profiles that would be useful for pulsar gating, scrunching, and binning.
The difference compared to normal mode is that the standard autocorrelations are turned off and instead are computed as if they are cross correlations.
This allows multiple pulsar bins to be stored. 
No formal cross correlations are performed. 
To be useful, one must create and specify a {\tt .binconfig} file and select only the pulsar(s) from the experiment.

See \url{http://cira.ivec.org/dokuwiki/doku.php/difx/vex2difx} for more complete information and examples.








% .xcb ------------------------------------------------------------------------

\subsection{.xcb} \label{sec:xcb}

\vspace{-20pt}\hspace{60pt}
\difxoneone

\vspace{7pt}

\noindent
When generation of sniffer output files is not disabled, each {\tt .FITS} file written by {\tt difx2fits} will be accompanied by a corresponding {\tt .xcb} file. 
This file contains cross-correlation spectra for each antenna for each baseline.
In order to minimize the output data size, spectra for the same source will only be repeated once per 15 minutes.
The file contains many concatenated records.
Each record has the spectra for all baseband channels for a particular baseline and has the following format which is very similar to that of the {\tt .acb} files. 
Note that no spaces are allowed within any field.
Values in {\tt typewriter} font without comments are explicit strings that are required.

\begin{center}
\begin{tabular}{l l l l}
\hline
Line(s) & Value & Units & Comments \\
\hline
1 & {\tt timerange:}          &                    & \\
  & {\it MJD}                 & integer $\ge 1$    & MJD day number corresponding to line \\
  & {\it start time}          & string             & e.g., {\tt 13h34m22.6s} \\
  & {\it stop time}           & string             & e.g., {\tt 13h34m52.0s} \\
  & {\tt obscode:}            &                    & \\
  & {\it observe code}        & string             & e.g., {\tt MT831} \\
  & {\tt chans:}              &                    & \\
  & {\it n}$_{\mathrm{chan}}$ & $\ge 1$            & number of channels per baseband channel \\
  & {\tt x}                   &                    & \\
  & {\it n}$_{\mathrm{BBC}}$  & $\ge 1$            & number of baseband channels \\
\hline
2 & {\tt source:}             &                    & \\
  & {\it source name}         & string             & e.g., {\tt 0316+413} \\
  & {\tt bandw:}              &                    & \\
  & {\it bandwidth}           & MHz                & baseband channel bandwidth \\
  & {\tt MHz}                 &                    & \\
\hline
3 to 2+$n_{\mathrm{BBC}}$ 
  & {\tt bandfreq:}           &                    & \\
  & {\it frequency}           & GHz                & band edge (SSLO) frequency of baseband channel \\
  & {\tt GHz polar:}          &                    & \\
  & {\it polarization}        & 2 chars            & e.g., {\tt RR} or {\tt LL} \\
  & {\tt side:}               &                    & \\
  & {\it sideband}            & {\tt U} or {\tt L} & for upper or lower sideband \\
  & {\tt bbchan:}             &                    & \\
  & {\it bbc}                 & {\tt 0}            & Currently not used but needed for conformity \\
\hline
3+$n_{\mathrm{BBC}}$ to 
  & {\it ant1 number}    & $\ge 1$            & number of first antenna \\
2+$n_{\mathrm{BBC}}(n_{\mathrm{chan}}+1)$  
  & {\it ant2 number}    & $\ge 1$            & \\
  & {\it ant1 name}      & string             & \\
  & {\it ant2 name}      & string             & \\
  & {\it channel number}      & $\ge 1$            & $= \mathrm{chan} + (\mathrm{bbc}-1) \cdot n_{\mathrm{chan}}$ for chan, bbc $\ge 1$ \\
  & {\it amplitude}           & $\ge 0.0$          & \\
  & {\it phase}               & degrees            & \\
\hline
\end{tabular}
\end{center}

\noindent
The above are repeated for each cross correlation spectrum record.
This file can be plotted directly with {\tt plotbp} or handled more automatically with {\tt difxsniff}.








% .ved .skd .vex.obs .skd.obs -------------------------------------------------

\subsection{.vex, .skd, .vex.obs, \& .skd.obs} \label{sec:dotvex}

\vspace{-20pt}\hspace{225pt}
\difxonefive

\vspace{7pt}

\noindent
The vex (Vlbi EXperiment) file \cite{vex} format is a standard observation description format used globally for scheduling observations and for driving the correlation thereof.
The original vex file for an experiment is typically created by {\tt sched} or {\tt sked}. 
In the former case (the case used by most astronomical VLBI), the vex file has the unfortunate file extension {\tt .skd}; in the later, the file extension is usually the less confusing {\tt .vex} .
These two vex formatted files contain only observation-scheduling based information.
A small amount of information based on the actualities of the observation are added by {\tt db2vex}, producing a new vex file with an additional file extension {\tt .obs} .
Please see vex documentation external to this manual for more information.
