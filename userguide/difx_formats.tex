\section{Baseband Data Formats}

\subsection{LBA} \label{sec:lbaformat}

\subsection{VLBA} \label{sec:vlbaformat}

\subsection{Mark IV} \label{sec:markivformat}

\subsection{Mark5B} \label{sec:mark5bformat}

\subsection{VDIF} \label{sec:vdifformat}

The VLBI Data Interchange Format (FIXME: reference to \url{vlbi.org}) is a very flexible baseband data format being used in almost all new VLBI backends (in 2015).
The variety of modes supported is very large.
As of DiFX version 2.5 just about all possible modes are supported in one way or another.

Some special VDIF concepts are described below.

\subsubsection{Fanout modes} \label{sec:vdiffanout}

The concept of fanout applies to certain variants of VDIF data where one logical sampled channel is interleaved multiple threads.
Certain modes of the DBBC3 makes use of this mode.
Reassembly of sampled data streams is possible using exactly the same mechanism as multiplexing multiple threads into a multi-channel, single-thread file.
Certain restrictions and usage rules apply:
\begin{enumerate}
\item Total number of threads is a multiple of the fanout factor.
\item The first {\em f} threads listed belong to the first channel to be reconstructed; these will become ``channel 0'' of the output single-thread VDIF file.
\item It is not possible to make use of fanout mode on VDIF data that contains multiple channels within each thread.
\end{enumerate}

\subsubsection{Extended Data Versions} \label{sec:vdifedv}

\subsubsection{Extended Data Version 4} \label{sec:vdifedv4}

